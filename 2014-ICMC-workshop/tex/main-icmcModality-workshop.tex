\documentclass{article}
\usepackage{icmcsmc2014}
\usepackage{times}
\usepackage{ifpdf}
\usepackage[english]{babel}
%\usepackage{cite}
\usepackage{fancyvrb}
%%%%%%%%%%%%%%%%%%%%%%%% Some useful packages %%%%%%%%%%%%%%%%%%%%%%%%%%%%%%%
%%%%%%%%%%%%%%%%%%%%%%%% See related documentation %%%%%%%%%%%%%%%%%%%%%%%%%%
%\usepackage{amsmath} % popular packages from Am. Math. Soc. Please use the 
%\usepackage{amssymb} % related math environments (split, subequation, cases,
%\usepackage{amsfonts}% multline, etc.)
%\usepackage{bm}      % Bold Math package, defines the command \bf{}
%\usepackage{paralist}% extended list environments
%%subfig.sty is the modern replacement for subfigure.sty. However, subfig.sty 
%%requires and automatically loads caption.sty which overrides class handling 
%%of captions. To prevent this problem, preload caption.sty with caption=false 
%\usepackage[caption=false]{caption}
%\usepackage[font=footnotesize]{subfig}


%user defined variables
\def\papertitle{Modality Workshop}
\def\firstauthor{First author}
\def\secondauthor{Second author}
\def\thirdauthor{Third author}


% workshop leaders so far:
% Till

% adds the automatic
% Saves a lot of ouptut space in PDF... after conversion with the distiller
% Delete if you cannot get PS fonts working on your system.

% pdf-tex settings: detect automatically if run by latex or pdflatex
\newif\ifpdf
\ifx\pdfoutput\relax
\else
   \ifcase\pdfoutput
      \pdffalse
   \else
      \pdftrue
\fi

\ifpdf % compiling with pdflatex
  \usepackage[pdftex,
    pdftitle={\papertitle},
    pdfauthor={\firstauthor, \secondauthor, \thirdauthor},
    bookmarksnumbered, % use section numbers with bookmarks
    pdfstartview=XYZ % start with zoom=100% instead of full screen; 
                     % especially useful if working with a big screen :-)
   ]{hyperref}
  %\pdfcompresslevel=9

  \usepackage[pdftex]{graphicx}
  % declare the path(s) where your graphic files are and their extensions so 
  %you won't have to specify these with every instance of \includegraphics
  \graphicspath{{./figures/}}
  \DeclareGraphicsExtensions{.pdf,.jpeg,.png}

  \usepackage[figure,table]{hypcap}
\fi

%setup the hyperref package - make the links black without a surrounding frame
\hypersetup{
    colorlinks,%
    citecolor=black,%
    filecolor=black,%
    linkcolor=black,%
    urlcolor=black
}


% Title.
% ------
\title{\papertitle}

% Authors
% Please note that submissions are NOT anonymous, therefore 
% authors' names have to be VISIBLE in your manuscript. 
%
% Single address
% To use with only one author or several with the same address
% ---------------
%\oneauthor
%   {\firstauthor} {Affiliation1 \\ %
%     {\tt \href{mailto:author1@smcnetwork.org}{author1@smcnetwork.org}}}

%Two addresses
%--------------
% \twoauthors
%   {\firstauthor} {Affiliation1 \\ %
%     {\tt \href{mailto:author1@smcnetwork.org}{author1@smcnetwork.org}}}
%   {\secondauthor} {Affiliation2 \\ %
%     {\tt \href{mailto:author2@smcnetwork.org}{author2@smcnetwork.org}}}

% Three addresses
% --------------
 \threeauthors
   {\firstauthor} {Affiliation1 \\ %
     {\tt \href{mailto:author1@smcnetwork.org}{author1@smcnetwork.org}}}
   {\secondauthor} {Affiliation2 \\ %
     {\tt \href{mailto:author2@smcnetwork.org}{author2@smcnetwork.org}}}
   {\thirdauthor} { Affiliation3 \\ %
     {\tt \href{mailto:author3@smcnetwork.org}{author3@smcnetwork.org}}}


% ***************************************** the document starts here ***************
\begin{document}
%
\capstartfalse
\maketitle
\capstarttrue
%
\begin{abstract}
Modality is a toolkit to improve and facilitate the use of digital technology within sound art and music, based on the audio programming language SuperCollider.
\end{abstract}

\begin{figure}[h]
	\centering
		\includegraphics[width=.9\columnwidth]{../media/20140403-IMG_1667.jpg}
	\caption{Public workshop and open Lab at STEIM}
	\label{fig:media_20140331-IMG_5976}
\end{figure}






\section{Overview of the modality concept and its aims}
\label{sec:overview_of_modality_concept_and_aims}

Modality is a project dedicated to modal interaction with synthesis processes for physical control in performance. Its primary product is the Modality Toolkit, a library to facilitate straightforward access to hardware controllers in the SuperCollider programming language. It is designed and developed by the ModalityTeam, a group of people that see themselves as both users and developers both of and for SuperCollider.

The idea behind the Modality Toolkit is to simplify the creation of individual electronic instruments using controllers of various kinds. 
To this end, a common code interface, MKtl, is used for connecting controllers from various sources and protocols. 
Currently HID and MIDI are supported with OSC, GUI-based interfaces can be created on the fly from interface descriptions.


\section{Scope of Modality}
\label{sec:scope_of_modality_tech_info_where_what}

\emph{Tells about the scope of the modality group, how it was established and how the meetings work. Should especially include a description of the combination of developer meeting, public workshop and concert.}

The name Modality arose from the idea to scaffold the creation of modal interfaces, i.e., to create interfaces where e.g. one physical controller can be used for different purposes or it is possible to switch its functionality, even at runtime. It is our belief that integration of such on-the-fly remapping features helps to create instruments that are flexible, powerful, and interesting to play. The strength of such a modal interface is that it allows for fast changes and more opportunity for sonic discovery as can be necessary when, for example, improvising with musicians playing acoustic instruments. 


The Modality toolkit aims to facilitate the following:

\begin{itemize}
	\item Access to data coming from commercial controllers, such as game controllers (human input devices; HID), MIDI controllers, OSC controllers (including mobile phone Apps), and other controllers, by providing a common interface to access these controllers. This includes creating specification files which characterize these devices.
	\item Access to data coming from custom/DIY controllers/interfaces, such as Arduino (or other microcontroller) based sensor interfaces, linking into the same common interface as above.
	\item Send output to these same controllers/interfaces.
	\item Graphical feedback of the current state of a device, as well as a possibility to replace a controller with a graphical user interface (GUI) substitute.
	\item Manipulation of data streams coming from these devices.
	\item Mapping the output of these data streams to output parameters of sound units.
\end{itemize}




Specific attention is given to the concept of modal control: the ability to ‘on -the-­fly’ i.e., within a performance, change the mapping of one control of a device to a control on another device, or change the functionality of one control based on the state of another control (compare to changing the function of an alphanumerical key by using the SHIFT key on a computer keyboard).


\section{Workshop content}
\label{sec:workshop_content}

The concept of the workshop is to be as “hands-­on” as possible.
Participants will learn how to use the Modality toolkit, create their own instruments, and play together with their creations. 
Under the guidance of the Modality Work Group, each participant will be able to work on his or her own instrument, and use and play it along with the other participants.

In detail, the outline of the workshop is as follows:

\begin{itemize}
	\item brief introduction to the Modality toolkit,
	\item installation party,
	\item mapping out devices (i.e., writing description files for them), in case the brought devices are not yet specified in Modality (contributing to the library of known devices),
	\item accessing devices and using data from devices; creating a basic sound--controller setup,
	\item performing with the system,
	\item swap controllers and share sounds between participants,
	\item adaption of the setup and manipulation of data streams,
	\item performing with the system
\end{itemize}



Participants should bring 
\begin{itemize}
	\item MIDI or HID controllers such as game gads, joysticks or fader boxes, 
	\item their own laptops with a recent version of SuperCollider installed, and
	\item a basic familiarity with the programming environment SuperCollider.
\end{itemize}


\begin{acknowledgments}
The Modality team is (in alphabetical order):
    Marije Baalman,
	Tim Blechmann,
    Till Bovermann,
    Alberto de Campo,
    Jeff Carey,
    Bjoernar Habbestad,
	Dominik Hildebrand Marques Lopes,
	Amelie Hinrichsen,
    Robert van Heumen,
    Hannes Hoelzl,
    Miguel Negrao, and
    Wouter Snoei.
Associated organisations are (in alphabetical order):
BEK,
the project \emph{Design, Development and Dissemination of New Musical Instruments} of UdK Berlin/TU Berlin, supported by the Einstein Foundation,
nescivi supported by theCreative Industry Fund NL, and
STEIM.

\end{acknowledgments} 

%%%%%%%%%%%%%%%%%%%%%%%%%%%%%%%%%%%%%%%%%%%%%%%%%%%%%%%%%%%%%%%%%%%%%%%%%%%%%
%bibliography here
\bibliography{smacsmc2014template}

\end{document}
