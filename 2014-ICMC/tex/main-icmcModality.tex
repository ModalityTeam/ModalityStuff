\documentclass{article}
\usepackage{icmcsmc2014}
\usepackage{times}
\usepackage{ifpdf}
\usepackage[english]{babel}
%\usepackage{cite}

%%%%%%%%%%%%%%%%%%%%%%%% Some useful packages %%%%%%%%%%%%%%%%%%%%%%%%%%%%%%%
%%%%%%%%%%%%%%%%%%%%%%%% See related documentation %%%%%%%%%%%%%%%%%%%%%%%%%%
%\usepackage{amsmath} % popular packages from Am. Math. Soc. Please use the 
%\usepackage{amssymb} % related math environments (split, subequation, cases,
%\usepackage{amsfonts}% multline, etc.)
%\usepackage{bm}      % Bold Math package, defines the command \bf{}
%\usepackage{paralist}% extended list environments
%%subfig.sty is the modern replacement for subfigure.sty. However, subfig.sty 
%%requires and automatically loads caption.sty which overrides class handling 
%%of captions. To prevent this problem, preload caption.sty with caption=false 
%\usepackage[caption=false]{caption}
%\usepackage[font=footnotesize]{subfig}


%user defined variables
\def\papertitle{Advances in Modality}
\def\firstauthor{First author}
\def\secondauthor{Second author}
\def\thirdauthor{Third author}

% adds the automatic
% Saves a lot of ouptut space in PDF... after conversion with the distiller
% Delete if you cannot get PS fonts working on your system.

% pdf-tex settings: detect automatically if run by latex or pdflatex
\newif\ifpdf
\ifx\pdfoutput\relax
\else
   \ifcase\pdfoutput
      \pdffalse
   \else
      \pdftrue
\fi

\ifpdf % compiling with pdflatex
  \usepackage[pdftex,
    pdftitle={\papertitle},
    pdfauthor={\firstauthor, \secondauthor, \thirdauthor},
    bookmarksnumbered, % use section numbers with bookmarks
    pdfstartview=XYZ % start with zoom=100% instead of full screen; 
                     % especially useful if working with a big screen :-)
   ]{hyperref}
  %\pdfcompresslevel=9

  \usepackage[pdftex]{graphicx}
  % declare the path(s) where your graphic files are and their extensions so 
  %you won't have to specify these with every instance of \includegraphics
  \graphicspath{{./figures/}}
  \DeclareGraphicsExtensions{.pdf,.jpeg,.png}

  \usepackage[figure,table]{hypcap}
\fi

%setup the hyperref package - make the links black without a surrounding frame
\hypersetup{
    colorlinks,%
    citecolor=black,%
    filecolor=black,%
    linkcolor=black,%
    urlcolor=black
}


% Title.
% ------
\title{\papertitle}

% Authors
% Please note that submissions are NOT anonymous, therefore 
% authors' names have to be VISIBLE in your manuscript. 
%
% Single address
% To use with only one author or several with the same address
% ---------------
%\oneauthor
%   {\firstauthor} {Affiliation1 \\ %
%     {\tt \href{mailto:author1@smcnetwork.org}{author1@smcnetwork.org}}}

%Two addresses
%--------------
% \twoauthors
%   {\firstauthor} {Affiliation1 \\ %
%     {\tt \href{mailto:author1@smcnetwork.org}{author1@smcnetwork.org}}}
%   {\secondauthor} {Affiliation2 \\ %
%     {\tt \href{mailto:author2@smcnetwork.org}{author2@smcnetwork.org}}}

% Three addresses
% --------------
 \threeauthors
   {\firstauthor} {Affiliation1 \\ %
     {\tt \href{mailto:author1@smcnetwork.org}{author1@smcnetwork.org}}}
   {\secondauthor} {Affiliation2 \\ %
     {\tt \href{mailto:author2@smcnetwork.org}{author2@smcnetwork.org}}}
   {\thirdauthor} { Affiliation3 \\ %
     {\tt \href{mailto:author3@smcnetwork.org}{author3@smcnetwork.org}}}


% ***************************************** the document starts here ***************
\begin{document}
%
\capstartfalse
\maketitle
\capstarttrue
%
\begin{abstract}
\end{abstract}

\begin{figure}[h]
	\centering
		\includegraphics[width=.9\columnwidth]{../media/20140331-IMG_5976.jpg}
	\caption{last workshop in Amsterdam}
	\label{fig:media_20140331-IMG_5976}
\end{figure}

\section{Overview of Modality Concept and Aims}
\label{sec:overview_of_modality_concept_and_aims}

Modality is a project dedicated to modal interaction with synthesis processes for physical control in performance. Its primary product is the Modality Toolkit, a library to facilitate straightforward access to hardware controllers in the SuperCollider programming language. It is designed and developed by the ModalityTeam, a group of people that see themselves as both users and developers both of and for SuperCollider.

The central idea behind the Modality Toolkit is to simplify the creation of individual electronic instruments using controllers of various kinds. To this end, a common code interface, MKtl, is used for connecting controllers from various sources and protocols. Currently HID and MIDI are supported with OSC, serial port and GUI-based interfaces are planned to be integrated.

The name Modality arose from the idea to scaffold the creation of modal interfaces, i.e., to create interfaces where e.g. one physical controller can be used for different purposes or it is possible to switch its functionality, even at runtime. It is our belief that integration of such on-the-fly remapping features helps to create instruments that are flexible, powerful, and interesting to play. The strength of such a modal interface is that it allows for fast changes and more opportunity for sonic discovery as can be necessary when, for example, improvising with musicians playing acoustic instruments. 



\section{Scope of Modality}
\label{sec:scope_of_modality_tech_info_where_what}




\begin{itemize}
	\item additions/unifications in basic SC, 
	\item Modality toolkit + Various Mixed Things + closely related
	\item more related libs - FP/FRP, wslib, JITLibExtensions, 
		all of Marijes device related quarks, etc 
\end{itemize}

\begin{figure}[h]
	\centering
		\includegraphics[width=.9\columnwidth]{../media/20140403-IMG_1667.jpg}
	\caption{Workshop and open Lab at STEIM}
	\label{fig:media_20140331-IMG_5976}
\end{figure}

\section{Islands and Bridges, uniform protocols}
\label{sec:islands_and_bridges_uniform_protocols}

Uniform devices (MIDI, HID, OSC, GUI, Serial)
		with rich descriptions, hierarchical names for all elements
		FakeGUI for everything
	
	Uniform destinations:
		set messages, RelSet, SoftSet, // if Influx, influence
		
\section{Methods}
\label{sec:methods}

Transformer islands: 
		FRP, Influx, others 
		- process incoming events, pass on results to destinations.

\section{Examples / Use Cases}
\label{sec:examples_use_cases}


\subsection{MPD 18}
\label{sub:mpd_18}

\section{Conclusions}
\label{sec:conclusions}


Modality - it's THE THING, yo!


\begin{acknowledgments}
The Modality team is (in alphabetical order):
    Marije Baalman,
	Tim Blechmann,
    Till Bovermann,
    Alberto de Campo,
    Jeff Carey,
    Bjoernar Habbestad,
	Dominik Hildebrand Marques Lopes,
	Amelie Hinrichsen,
    Robert van Heumen,
    Hannes Hoelzl,
    Miguel Negrao, and
    Wouter Snoei.
Associated organisations are (in alphabetical order):
BEK,
the project \emph{Design, Development and Dissemination of New Musical Instruments} of UdK Berlin/TU Berlin, supported by the Einstein Foundation,
nescivi, and
STEIM.

\end{acknowledgments} 

%%%%%%%%%%%%%%%%%%%%%%%%%%%%%%%%%%%%%%%%%%%%%%%%%%%%%%%%%%%%%%%%%%%%%%%%%%%%%
%bibliography here
\bibliography{smacsmc2014template}

\end{document}
